\documentclass{article}

\begin{document}

\noindent
\Huge
\sloppy
Effects and Application of Force Feedback within Virtual Reality \\ \\
%\newline
\normalsize
\textbf{Nicholas Muszynski} \\
University of Massachusetts Lowell, Computer Graphics II \\
nmsznsk@gmail.com \\
November 27, 2017 \\ \\

\noindent
\textbf{Abstract:} This paper will cover the impact of haptic feedback within virtual reality, such as belief of realism or 'presence', and review both standard and experimental methods of application of haptic feedback to achieve greater presence and other results. \\ \\
--------------------------------------------------------------------------------- \\ \\

\textbf{1 - Overview} \\
The Somatosensory system, also known as the sense of touch, is an important component in the human body for receiving information on whether  an object is hot or cold, hard or soft, wet or dry, which helps humans understand the world around them. Because our sense of touch is so important and our development and enjoyment of media continues to advance, adding a sense of touch to how we experience media seems like a natural next step to take. In doing so, force feedback systems (henceforth referred to as 'haptics') were developed to increase a user's immersion, experience, and enjoyment with all sorts of applications, from Disney's 3D rides to vibrating Rumble Pak attachments for Nintendo 64 controllers. Vibrating controllers for game consoles are now an industry standard along with other haptic implementations, but a particular industry where this has yet to become standardized is within Virtual Reality systems. Since one of the primary reasons for virtual reality is to enhance presence (which is a user's belief that events occurring within virtual reality are actually taking place), haptics is an area of development that is being focused on by various companies, as adding the sense of touch to the current mix of visual, audible, and kinesthetic (movement) sensations will further increase a user's experience within virtual reality, helping them to "feel like they're really there". This also has the added benefit of reducing a user's proneness to virtual reality-induced nausea, which typically occurs from the jarring effect of our eyes perceiving an occurring event that the rest of our senses don't agree with, like physically standing still while our eyes perceive ourselves moving in a certain direction within virtual reality. \\ \\

\textbf{2 - Effects and Application} \\
Since its conception, haptics have become a standard for games and other media due to its ability to add a new dimension to how users experience the media. Likewise, haptic controls and feedback have the potential to greatly improve a user's sense of presence within virtual reality. This helps accomplish one of the primary attractions of virtual reality where the user can believe they are within the environment they are experiencing, and also assists in reducing one of the primary disadvantages of virtual reality, which is the possibility of experiencing virtual reality-induced nausea, by enhancing a user's sense of presence. As such, both of these resulting effects are strong justifications for developing and enhancing force feedback and touch controls into more advanced systems, allowing users to both touch and be touched by virtual reality. \\
Currently, there are rudimentary haptic designs already implemented within popular virtual reality systems, such as the Oculus Rift\cite{article1cite} and the HTC Vive\cite{article2cite}, in the form of wand-type controllers. These controllers come in pairs, one for each hand, and feature motion tracking within virtual reality, force feedback through vibration within the controller, and buttons on the controller to help determine what shape a user's hand is making. For example, the Oculus Rift's Touch controllers can determine what gesture a user makes with their thumb and index finger, such as a thumbs-up, pointing, a closed fist, and any combination in between\cite{article1cite}. Software developers can use these features to enhance a user's experience within virtual reality: instead of pressing a button on a controller when a user is near an object to pick it up, they now have to move their hand to the location of the object, then pick it up by making a fist within virtual reality, which is done by enclosing their hand on the controller, naturally depressing triggers which will send input to the software that the user is closing their hand, and the user will see their virtual hand turn into a fist accordingly. The object could then be manipulated by the user, such as picking up an apple and moving the object close to the user's face and rotating it to examine the apple further, throwing the apple, dropping it, or setting it back down. Haptics are also available in the form of vibrating motors within the controllers, which developers can use to implement force feedback whenever such interactions are made, providing the illusion that a user's hand is meeting a small amount of resistance when they interact with an object. \\
Advancements continue to be made for touch-based controls, whether improving upon the current design of wand controllers by implementing methods to receive input from all five fingers such as Valve's Knuckles controller\cite{article3cite}, or new methods of touch controls such as the hand tracking technology of Leap Motion\cite{article4cite} which utilizes two cameras and three infrared LEDs\cite{article4cite} to compile an image that is processed by software to determine the position and shape of a user's hands, and glove controllers like the Manus-VR Gloves\cite{article5cite} which have various sensors built into the gloves and each finger, along with force feedback vibration motors built into the gloves, to provide accurate tracking of the shape, position, and motion of a user's hands. However, hand-based controllers are not the only area of focus for haptics; full bodied suits are being developed such as the Teslasuit\cite{article6cite}, which provide various force feedback features in the base form of electrical pulses of varying amplitude, frequency, and amperage to specific points on the suit\cite{article6cite}. These pulses can be adjusted to provide appropriate situations to what may be occurring within virtual reality, such as an object being thrown at a user's chest, something tapping a user on the back, replicate the sensation of an area feeling hot or cold, and give the impression of weight to an object that might be held. These suits also come equipped with positioning systems for motion tracking much like the hand-based controllers, but across the entire body instead. \\ \\

\textbf{3 - Summary and Review} \\
Possibly only several years from now, today's technology of full-body haptic suits and complex hand controllers will likely become standard equipment for virtual reality systems, while tomorrow's technology will push to improve upon current methods and innovate new designs. These concepts may also be applied to other fields and goals such as rehabilitation and training, and not exclusively to the realm of virtual reality. One possible method of functionality to implement for virtual reality could be a marching-in-place motion, which has been researched\cite{article7cite} and found to reduce VR-sickness in users, due to the marching-in-place motion increasing a user's sense of presence from a software using this motion as movement input to move the player forward within virtual reality. So far this has been implemented by using equipment attached to a user's legs which will sense the up-down movement of this action and alternating between legs, but with suits such as the Teslasuit, motion tracking could be utilized instead of motion sensing hardware to register a user's leg movement so that no extra hardware needs to be worn. Another potential area to explore would be the sense of physical resistance based on interactions within virtual reality. For example, if a user were to collide with a wall in virtual reality, an effect would trigger which may inhibit a user's ability to move their body in such a way that would intersect through the wall. This may potentially be accomplished through electrical pulses inhibiting a user's muscles from contributing further to any such intersection, however this would likely be painful to experience and raise other safety concerns. For this area of focus, we may simply have to settle for heavy force feedback from collision within virtual reality, but it becomes apparent just how crucial haptics can and will be in adding a new dimension to our ability to experience virtual reality.

\bibliography{articles}
\bibliographystyle{ieeetr}

\end{document}