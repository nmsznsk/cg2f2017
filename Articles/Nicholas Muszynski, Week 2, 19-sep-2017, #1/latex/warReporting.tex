\documentclass{article}
\begin{document}



\title{Can Virtual Reality Change How People Respond to War Reporting?}
\author{Nicholas Muszynski}
\date{December 14, 2017}

\maketitle

\textbf{Summary}
\newline

In this article, a war correspondent who focuses on photography would normally display their photographs of war in exhibits, with the hope of having participants question their perspective of conflicts in an attempt to encourage peace. However, the correspondent was worried that photographs on their own didn't reflect the experience of these conflicts first-hand, and decided to try implementing an exhibit with virtual reality. One such exhibit was set up in MIT, and participants would wear a backpack with a laptop inside, and an Oculus Rift headset. Since the Oculus Rift isn't an augmented reality headset and completely covers your eyes, I'm assuming the entire exhibit was rigged with cameras and would track the participants as they walked around, but it wasn't mentioned how many people were allowed into the exhibit at once. Within the exhibit, portaits of various photographs would fade into and out of existence on the walls, and there were also full renderings of fighters from both sides of various conflicts which would move and speak naturally, with dialogue from various questions that the correspondent had asked the fighter. By using virtual reality, the experience seems to leave a much stronger impact than photography, and the correspondent hopes to bring this exhibit to the locations where the conflicts are occurring in an effort to promote peace.

\nocite{warReportingCite}

\bibliography{warReporting}
\bibliographystyle{ieeetr}

\end{document}