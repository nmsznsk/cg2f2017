\documentclass{article}
\begin{document}



\title{Augmented & Virtual Reality: The Future of Work, Not Just Play}
\author{Nicholas Muszynski}
\date{December 14, 2017}

\maketitle

\textbf{Summary}
\newline

This article discusses the use of virtual reality in the workplace in the form of safety training so that an organization's employees will have more experience for a issues that may occur, and will be more familiar with how to respond in the situation. As a result, companies can save money from reparations of these scenarios by investing in preventing them from occurring in the first place with more in-depth training for employees. Companies were also saving money simply by using virtual reality as an alternate training method to what was previously the standard method. Another aspect of safety that is improved is by providing employees with training experience through virtual reality in regards to their profession, such as aviation or medical professions. The article also mentions the use of live video feeds so that an expert can see a real-time situation that a novice may be experiencing, but this doesn't really seem relevant to augmented or virtual reality as this would be categorized as video streaming instead, unless there is some functionality for the expert to point out or otherwise notify the novice visually of something they may need to do.

\nocite{VRforWorkCite}

\bibliography{VRforWork}
\bibliographystyle{ieeetr}

\end{document}