\documentclass{article}
\begin{document}



\title{Emotional Qualities of VR Space}
\author{Nicholas Muszynski}
\date{December 15, 2017}

\maketitle

\textbf{Summary}
\newline

This article reviews whether the emotional reaction a person has to light, color, and texture, within an area can be replicated within virtual reality. The goal of this test is to determine if virtual reality could be utilized to further enhance an architect's design process. To test this, they would switch the environment furnishings between blue/orange, light/dark, and rough/smooth respectively for each parameter, and had users either do nothing or perform a basic activity such as folding clothes. The virtual environment itself was displayed within the Cave Automatic Virtual Environment (CAVE) system, which is a room that users can stand in and have objects virtually generated within through the use of specialized eyewear. The test was successful in causing users to feel that an area was conceptually much warmer and cooler based on altering the parameters noted above. While the data on its own can be utilized by a home designer, these same concepts can also be implemented into virtual reality to induce the same responses in individuals and achieve a desired result.

\nocite{emotionCite}

\bibliography{emotion}
\bibliographystyle{ieeetr}

\end{document}