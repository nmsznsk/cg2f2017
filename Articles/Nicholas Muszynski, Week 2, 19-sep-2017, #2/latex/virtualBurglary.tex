\documentclass{article}
\begin{document}



\title{Virtual Burglary: Exploring the Potential of Virtual Reality to Study Burglary in Action}
\author{Nicholas Muszynski}
\date{December 14, 2017}

\maketitle

\textbf{Summary}
\newline

In this article, virtual reality is explored as a potential method for studying illegal activity such as burglaries, since the nature of the subject is very difficult to study extensively in real environments. By utilizing virtual reality, studies can be performed that closely examine the burglar themselves to get a better idea of how an individual may make in-the-moment decisions in various burglary situations. This would also help determine what aspects of a burglar's personality may have an impact on the decision-making committing a crime. The user would wear a VR headset and use a controller to move around and explore neighborhoods with well-detailed interiors and exteriors. They would then be instructed to target a particular house with various valuable objects within. The moment the user enters the house, a timer starts counting down, and once three minutes pass an alarm sounds. If the user isn't out of the house 3 more minutes after the alarm sounds, then they are notified as being caught. The results of the study shows that virtual reality appears to be a promising approach to examining criminal activities, as users reported feeling a high level of presence during the simulation which is further confirmed by physiological data, and the results seem to correlate with already-known behavior patterns. With further research, virtual reality can help develop more effective preventative measures against burglary and other crimes in the future.

\nocite{virtualBurglaryCite}

\bibliography{virtualBurglary}
\bibliographystyle{ieeetr}

\end{document}