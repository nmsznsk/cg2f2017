\documentclass{article}
\begin{document}



\title{Maroon VR: A Room-scale Physics Laboratory Experience}
\author{Nicholas Muszynski}
\date{December 15, 2017}

\maketitle

\textbf{Summary}
\newline

This article explores the use of virtual reality in physics education, as virtual laboratories allow students to safely experience and test scenarios in an easily-digestible and intuitive fashion, as well as better engage students and promote their interest in the subject matter. Maroon VR was created to accomplish this goal by allowing students to use the HTC Vive system to perform experiments and learn about various concepts. For testing purposes, the focus was on electromagnetism, and utilized various experiments within the application to allow students to experiment with this topic. Results of the test were positive, with majority rating the experience as enjoyable and motivating, along with making it easier to understand the concepts behind the subject matter. This application of virtual reality could potentially be brought to other subjects as well, such as history, and promote better learning results.

\nocite{physicsCite}

\bibliography{physics}
\bibliographystyle{ieeetr}

\end{document}