\documentclass{article}
\begin{document}



\title{VR-based Remote Control System for Rescue Detection Robot in Coal Mine}
\author{Nicholas Muszynski}
\date{December 15, 2017}

\maketitle

\textbf{Summary}
\newline

This article focuses on utilizing virtual reality to control robots which are designed to explore and investigate mines that are considered dangerous or experienced an environmental disaster. This is accomplished by building a virtual layout based on data sent from the robot's sensors, so that a copy of the robot's environment is created and can be viewed in safety. This also allows the user to effectively control the robot based on the map built within virtual reality, because they will have a much better idea of the surroundings and determine where to go next. The data is used to create point cloud data, and while it isn't mentioned whether or not previous areas will remain on screen, this could be used as an effective mapping tool if there isn't any loss of GPS tracking, enabling users to not only explore but potentially map the caves in an extensive manner. There didn't seem to be any testing of whether or not holes in the ground or other hazards would be effectively mapped, but this can prove to be an effective way of safely exploring cave systems in the future.

\nocite{robotCite}

\bibliography{robot}
\bibliographystyle{ieeetr}

\end{document}