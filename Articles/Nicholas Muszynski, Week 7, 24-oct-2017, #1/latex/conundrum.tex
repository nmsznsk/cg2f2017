\documentclass{article}
\begin{document}



\title{Slate's Virtual Conundrum}
\author{Nicholas Muszynski}
\date{December 14, 2017}

\maketitle

\textbf{Summary}
\newline

This article goes into detail about a Facebook chat show which uses virtual reality to show the host and any guests they may have for the day. Their show utilizes a Facebook app called Spaces, which allows friends to sit together in a virtual room and interact with each other. With the built-in features of Spaces, the show is able to display various environments in the form of 360 degree photos, and viewers see the discussion as if they are one of the members in the group chat, but this isn't explained within the article, however it seems to be the case based on the picture next to the title. In order to use Spaces, participants need to have an Oculus Rift; this also means that one needs a computer that is capable of handling the virtual reality system. This is also another aspect that isn't covered within the article, but it is likely that not all of their guests have an Oculus Rift, let alone a computer powerful enough to handle it. I would guess that the team likely sends over the hardware to a guest once an agreement is made to be on the show, and while this is likely quite a hassle, it's less of a hassle of getting on a plane to visit the studio for a day. Overall, a unique and interesting take on using virtual reality.

\nocite{conundrumCite}

\bibliography{conundrum}
\bibliographystyle{ieeetr}

\end{document}