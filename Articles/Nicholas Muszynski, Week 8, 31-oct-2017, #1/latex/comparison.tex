\documentclass{article}
\begin{document}



\title{A comparison between Oculus Rift and a low-cost smartphone VR Headset: Immersive user experience and learning}
\author{Nicholas Muszynski}
\date{December 15, 2017}

\maketitle

\textbf{Summary}
\newline

This article tests whether or not there may be a difference between a high-end (Oculus Rift, HTC Vive, etc.) versus a low-end (Gear VR, Google Cardboard, etc.) virtual reality system, in terms of education for students. The fear is that low-cost smart phone virtual reality systems may hamper a user's ability to learn due to limitations of capabilities of the hardware in comparison to higher end models. An application called "Titans of Space" was used since it could be run on both mobile and computer platforms, and is an educational tour of the solar system within virtual reality. Half of their participants used the mobile platform, and the other half used the computer platform. They were toured around the solar system, and were then asked to see what the order of the planets were from closest to farthest from the sun. From before testing to after testing, 5 more mobile users were correct (from 1 -> 6). For computer users, 6 more users were correct (from 4 -> 10). It was determined that the difference of 5 vs. 6 wasn't significant; as a result, mobile virtual reality systems can be considered a viable method of using virtual reality for education purposes, while also being a cost-efficient alternative to computer platforms.

\nocite{comparisonCite}

\bibliography{comparison}
\bibliographystyle{ieeetr}

\end{document}