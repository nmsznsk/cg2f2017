\documentclass{article}
\begin{document}



\title{Gravitational Pull: Game Artists use Scientific Data to Re-Create Mars for a Compelling VR Experience}
\author{Nicholas Muszynski}
\date{December 15, 2017}

\maketitle

\textbf{Summary}
\newline

This article discusses how a small team of game developers collaborated with NASA in an attempt to build an authentic virtual reality experience of visiting Mars. However, this isn't being approached as a game, but rather a training environment where everything within is as accurate as possible, including physics, sound, environments, visuals, and so on. For example, to generate the environments, they used height maps generated from satellite imaging and probing of the planet and generated the landscape within a deviation of one foot. Great detail went into the development of each asset; for example, the rocks used in the application are composed of between 2,000 to 3,000 polygons so that they are well detailed, with the scene itself sitting at approximately 53 million triangle, and the area is of the landscape is about 22 square kilometers. The project will be released for the Oculus Rift and HTC Vive systems once it's completed.

\nocite{marsCite}

\bibliography{mars}
\bibliographystyle{ieeetr}

\end{document}