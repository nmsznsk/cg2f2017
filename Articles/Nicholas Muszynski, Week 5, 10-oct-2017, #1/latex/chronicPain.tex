\documentclass{article}
\begin{document}



\title{The Impact of Virtual Reality on Chronic Pain}
\author{Nicholas Muszynski}
\date{December 14, 2017}

\maketitle

\textbf{Summary}
\newline

This article considers whether or not virtual reality may be a potential method for alleviating chronic pain. 
This is based on the idea of "Gate Theory", where a user's attention is diverted to something else so that they aren't focusing as much on their pain. Participants were given VR headsets and a mouse to interact with menus inside an application as they played it, for a duration of 5 minutes. The application itself is called "COOL!", and is designed to slowly take a user along a trail to see various landscapes (and otters). Overall, the study appears to be a success, as 27 out of 30 participants reported feeling less chronic pain during and after the 5 minute session. On a scale of 0 to 10, with 10 being extreme pain and 0 being no pain, users averaged around 5.7 before the session, 2.6 during the session and 4.1 after the session. This can prove to be an effective and safer (and likely cheaper) alternative to opiates which are sometimes used to treat chronic pain.

\nocite{chronicPainCite}

\bibliography{chronicPain}
\bibliographystyle{ieeetr}

\end{document}