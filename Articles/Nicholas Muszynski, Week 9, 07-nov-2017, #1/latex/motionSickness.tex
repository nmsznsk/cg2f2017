\documentclass{article}
\begin{document}



\title{360 Degree Stereo Image based VR Motion Sickness Testing System}
\author{Nicholas Muszynski}
\date{December 15, 2017}

\maketitle

\textbf{Summary}
\newline

This article review motion sickness inflicted by virtual reality, and attempts to limit the amount of sickness caused by looking for warning signs that it may occur. The approach is to look for several factors: rapid camera rotation, abnormal camera movement (suddenly going backwards very quickly), how much time elapses between movement, and so on. The test system will review an application frame-by-frame to determine if there is any potential for inducing sickness, and generates a report as a final result which will tell what time and what frame may be causing problems. This is quite different from the typical approach of users reporting when they may feel sick, as this program will scan the output of an application with image-evaluating functions to determine if there is any cause for concern.

\nocite{motionSicknessCite}

\bibliography{motionSickness}
\bibliographystyle{ieeetr}

\end{document}