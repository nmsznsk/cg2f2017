\documentclass{article}
\begin{document}



\title{Virtual Reality: The Next Frontier of Audiology}
\author{Nicholas Muszynski}
\date{December 14, 2017}

\maketitle

\textbf{Summary}
\newline

Audiology is the study of hearing and any disorders or disabilities that are related to it. This article considers the possibility of using virtual reality to replace ASL (American Sign Language) translators completely, as there are various situations where a translator may not be around or available. While smart phones are already capable of running applications which can effectively perform speech-to-text functionality, this isn't as necessarily useful as may be first believed since English may be a user's second language and ASL is their first. As mentioned in the article, a good analogy of this would be using a speech-to-text translator which is written in Spanish for someone where Spanish isn't a fluent language. An application is being designed which shows a virtual person that translates speech to ASL and other sign languages, and using correct facial expressions and hand motions to be as competent as someone who is fluent in it. The article also discusses the application of virtual reality in similar avenues, such as retraining a person's vestibular system to help reduce their chronic dizziness, and translating ASL into speech or text with the use of a motion tracking glove (although I wouln't personally consider this particularly VR-related).

\nocite{audiologyCite}

\bibliography{audiology}
\bibliographystyle{ieeetr}

\end{document}