\documentclass{article}
\begin{document}



\title{Cane Walk in the Virtual Reality Space using Virtual Haptic Sensing}
\author{Nicholas Muszynski}
\date{December 15, 2017}

\maketitle

\textbf{Summary}
\newline

This article explores the use of haptic feedback systems to allow visually impaired users to interact with virtual reality environments. The design works by having an object loaded into a virtual environment and a glove is worn and tracked with scanners, resulting in a virtual version of the the user's hand being loaded into the environment close to the object. The specialized glove has a haptic feedback device on each finger, which would give feedback when the user was touching an object within virtual reality.  By doing so, a visually impaired user would be able to identify the shape and details of the object based on this feedback, as if they were touching a physical object. Building on top of this, a real cane was used to touch a virtual padded block where the floor is, and additional haptic feedback was induced from the glove, giving the sensation to the user that there was something on the floor. The feedback itself was generated with electrical pulses, which varied in strength and duration based on the object that is interacted with. With further implementation, it may be possible to provide a satisfying virtual reality experience for virtually impaired users.

\nocite{visualCite}

\bibliography{visual}
\bibliographystyle{ieeetr}

\end{document}