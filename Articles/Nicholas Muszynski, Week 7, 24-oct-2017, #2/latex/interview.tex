\documentclass{article}
\begin{document}



\title{Virtual Reality Job Interview Training for Veterans with Posttraumatic Stress Disorder}
\author{Nicholas Muszynski}
\date{December 14, 2017}

\maketitle

\textbf{Summary}
\newline

This article discusses the use of virtual reality to help those with PTSD (Posttraumatic Stress Disorder) prepare for job interviews. Veterans with PTSD appear to experience symptoms which hinder their likelihood of being employed, such as fears of discussing their military history, social interaction, job history, and so on. By placing participants in a virtual interview (known as exposure therapy), they are able to practice the interview process and become experienced and more familiar with it, and help overcome their fears. Users reported that the application helped build their confidence and overall comfort of the interview process. With these results, it is likely that virtual reality can also be used in other scenarios to help acclimate people with various conditions to social interactions. There also seems to be no virtual reality headsets or other related hardware used in this study; only a simulation program was displayed on a screen. The entire process may likely have been improved by increasing immersion through the use of a virtual reality headset.

\nocite{interviewCite}

\bibliography{interview}
\bibliographystyle{ieeetr}

\end{document}