\documentclass{article}
\begin{document}



\title{Immersive and Collaborative Taichi Motion Learning  in Various VR Environments}
\author{Nicholas Muszynski}
\date{December 15, 2017}

\maketitle

\textbf{Summary}
\newline

This article proposes a system for learning an art or activity that is heavily focused on physical movement, such as Tai Chi within this particular example. By using virtual reality, the application captures a participant's movement and displays this as an avatar within the environment, which allows a student to be able to have a better perspective and understanding of the teacher's movements. Likewise, it can capture a student's movement as well so that the teacher can review and provide critique and guidance. Users practiced various movements via just a monitor, using a head mounted display, and using the CAVE system. In comparison to simply watching a video on a monitor, users reported that virtual reality provided a more realistic learning environment and learning was generally more efficient, but some drawbacks with using a head mounted display occurred where the quality of the students' movements weren't as acceptable, but this is believed to be due to the student being unable to view their own body within virtual reality. This application would also be useful for learning to dance, although the size limitation of the physical space that the practice is being performed in will likely cause issues.

\nocite{taichiCite}

\bibliography{taichi}
\bibliographystyle{ieeetr}

\end{document}