\documentclass{article}
\begin{document}



\title{EmeTerm Offers Solution to VR Motion Sickness }
\author{Nicholas Muszynski}
\date{October 3, 2017}

\maketitle

\textbf{Summary}
\newline

EmeTerm is a new product designed by WAT Med that is intended to reduce symptoms of nausea caused by motion sickness. It is a wristband, labeled as an "`anti-emetic device"', and while it is intended for any motion sickness induced by riding a car, boat, or plane (or even morning sickness caused by pregnancy), it is believed to potentially have the same effect on VR-induced sickness as well. The device is described as reducing the intensity of the input signal to the brain from scenarios such as riding on a boat or in a car, and also reducing the intensity of input signal to the stomach from the brain which triggers the stomach to contract and cause the person to vomit. The wristband had already received an iF Design Award for 2017, and if it actually reduces these symptoms, can be an attractive option for those who may be apprehensive of taking medicine due to potential side effects, among other benefits.

\nocite{article2cite}

\bibliography{article2}
\bibliographystyle{ieeetr}

\end{document}