\documentclass{article}
\begin{document}



\title{Fear in Virtual Reality}
\author{Nicholas Muszynski}
\date{September 26, 2017}

\maketitle

\textbf{Summary}
\newline


A study was done to determine what made VR horror games instill fear in players. 
There were two types of classifications of fear: Place Illusion, and Plausibility Illusion. 
Place Illusion is when a player believes or perceives that they are actually within the environment. 
Plausibility Illusion is when a player believes that an event that is occurring within the game is actually real. 
VR horror games appear to be more effective at inflicting these two types of illusion than movies or other media are capable of doing. 
As such, it may be recommended to use VR horror games as a possible treatment to phobias that a patient may be experiencing. 
Coping methods that players would use would mainly be verbal methods of emotional expression, such as self-talk, screaming, and swearing. 
Players would also try to disengage from the game by closing their eyes, telling themselves that it isn't real, and attempting to distract themselves by thinking of other things. 
It was also noted that there was a lasting effect of fear, with a sizable portion of players stating that they would experience a greater fear of the dark, among other heightened fears.


\nocite{article2cite}

\bibliography{article2}
\bibliographystyle{ieeetr}

\end{document}