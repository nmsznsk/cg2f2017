\documentclass{article}
\begin{document}



\title{A Study on Immersion and VR Sickness in Walking Interaction for Immersive Virtual Reality Applications}
\author{Nicholas Muszynski}
\date{September 26, 2017}

\maketitle

\textbf{Summary}
\newline

This article reviews the difference of "Presence" achieved for various methods of controlling movement within a Virtual Reality application. Three methods were tested, where the first reviewed the use of gamepad controllers for movement, which is currently the most popular method of movement within VR applications. The second and third methods utilized hand-based motions to control movement, and a march-in-place detection simulator which would register movement of the user's legs to control movement, respectively. After further testing, the results suggested that the more the user's movement mimicked actual movement for the relevant physical actions (running motion for running, etc.), the greater the "Presence" achieved for users, resulting in a more pleasant experience for users. Obtaining a higher perception of presence is a desirable result, as this lowers a user's risk of becoming afflicted with VR sickness, allowing them to user the application for longer periods of time, and being more inclined to use it again after they finish.

\nocite{article1cite}

\bibliography{article1}
\bibliographystyle{ieeetr}

\end{document}