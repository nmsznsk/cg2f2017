\documentclass{article}
\begin{document}



\title{Effects of 3D Virtual Haptics Force Feedback}
\author{Nicholas Muszynski}
\date{October 3, 2017}

\maketitle

\textbf{Summary}
\newline

This article focused on determining how much of an effect haptics had on a user's feeling of physical presence for virtual objects. Haptics is characterized as, "`The modality of touch and associated sensory feedback."', while physical presence is considered a state where virtual objects are viewed as being actual physical objects. The test used a system called the Novint Falcon, which was capable of performing 3-dimensional movement, and was utilized to play a racing game, where players could feel lateral movements from turns along with bumps on the road. Compared to users who had no feedback systems enabled, those who experienced haptic feedback reported much stronger senses of physical presence. It was also noted that users felt that the virtual car in the test without haptics was "`less rugged"' compared to the virtual car in the test with haptics enabled, which was considered to potentially lead to interesting studies within advertising.

\nocite{article1cite}

\bibliography{article1}
\bibliographystyle{ieeetr}

\end{document}