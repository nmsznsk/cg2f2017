\documentclass{article}

\begin{document}

\noindent
\Huge
\sloppy
Treatment and Prevention of Virtual Reality Induced Sickness \\ \\
%\newline
\normalsize
\textbf{Nicholas Muszynski} \\
University of Massachusetts Lowell, Computer Graphics II \\
nmsznsk@gmail.com \\
October 10th, 2017 \\ \\

\noindent
\textbf{Abstract:} This paper reviews various articles and research performed on Virtual Reality induced sickness, reviewing both the causes and any possible prevention and treatment methods that can be utilized to help reduce or eliminate VR sickness from occurring in users. \\ \\
--------------------------------------------------------------------------------- \\ \\
\textbf{1 - Overview} \\
As Virtual Reality becomes more popular and accessible as a form of entertainment and as a tool for training or rehabilitation, a major topic of concern for many users are the symptoms that they experience when using Virtual Reality hardware, with these symptoms being categorized as Virtual Reality induced sickness (henceforth referred to as 'VR Sickness'). While some users experience minimal symptoms or none at all, other users can be severely afflicted and cause them to become averse to using the hardware again, which can be a major hindrance as an avenue of enjoyment, or as a tool for recovery. As such, determining possible methods to treat VR Sickness or preventing it from occurring will potentially enable more users to utilize VR when they may not have been able or willing to do so before. Various articles were reviewed to analyze what causes VR Sickness to occur, as well as research different methods and techniques which may counter the cause of these symptoms by finding solutions in both hardware and software design. The results of these articles are then examined for similarities and differences, and what new research these results may guide us to in the future. \\ \\

\textbf{2 - Cause and Prevention} \\
In order to resolve the problem of VR Sickness, the cause of the symptoms needs to be examined and discussed first. The human body has multiple systems that work in tandem to govern how the body interprets and interacts with the environment around it. Of these systems, two are of particular importance in this issue: visual perception, and the vestibular system. Visual perception is the input our brain receives from our eyes, while the vestibular system governs our sense of balance and interpreting our physical movement within an environment, also providing the brain with input of this information. It is a combination of these two, and other, systems which allow us to move and interact in a coordinated manner; so what happens then if the input from these two systems start to conflict with each other? by exposing a person to mismatched input from their visual and vestibular perceptions, the person will potentially experience VR Sickness\cite{article4cite}. Put simply, it is the effect of "visual and sensory dissonance"\cite{article5cite} that causes users to experience symptoms such as nausea, vomiting, fatigue, disorientation, headaches, and more\cite{article3cite}. \\
An example of mismatched input can be given as a user who is using a VR headset while sitting at a desk, and playing a game which utilizes a first-person perspective and involves the impression of movement throughout a virtual environment. While the user's visual perception is interpreting the movement within the virtual environment, the vestibular system does not register any significant shift in balance, because the user is not physically moving when there is any movement within the game. As a result, a conflict of information occurs, where one input suggests movement, while the other does not. \\ \\
To gather more information, a study was done where users were intentionally exposed to a virtual environment built in such a way that it would exacerbate the mismatch of information between a user's visual perception and vestibular system\cite{article4cite}. The testing environment involved having a user stand in a room which had screens on each wall, and the user wore polarized glasses configured with an electromagnetic tracking system, allowing the user to view a virtual ball which would move around the room, encouraging the user to also move around within the environment to examine the ball. However, the environment was configured so that any movement would cause double the amount of distance traveled on the screens, such as turning around causing the walls on the screen to rotate twice as much. This caused the user to perceive much more movement occurring than what their vestibular system was registering, and inflicted VR Sickness on the user. These tests supported the idea that mismatching visual and vestibular data to the brain caused VR sickness\cite{article4cite}. \\ \\
While some studies were being performed to research further on the causes of VR Sickness, others were testing various situations and methods to determine what may counter it. One study focused on determining the increase of 'presence' from movement, which is the illusion of belief that a user is actually within a virtual environment, by comparing varying methods of movement within a VR simulation. Three types of movement were examined: movement input by gamepad, hand gestures, and a marching-in-place motion\cite{article1cite}. While gamepad controls and hand gestures offered little in the way of increasing a user's perception of presence and reducing the severity of VR Sickness as the users weren't using their legs to generate the movement, but performing a marching-in-place motion that was captured by electronic bands attached to their ankles appeared to have a noticeable impact on presence and VR Sickness\cite{article1cite}. Even though a user was only walking in place to achieve movement within the virtual environment, the act seems to have a positive impact against VR Sickness, helping to trick the body into believing actual movement is occurring. \\
Registering physical movement of the user isn't the only method being explored; there was also a study done to determine what effect the Field Of View within a virtual environment has on VR Sickness. The goal of this study was to subtly and dynamically adjust a user's Field Of View based on the movement of the user within the virtual environment by decreasing the Field Of View when they were moving, and increasing when they were standing still, as decreasing a user's Field Of View appears to decrease the severity of VR Sickness at the cost of also decreasing presence\cite{article3cite}. In doing so, by dynamically reducing the Field Of View when a user is moving, which is when they are more likely to be prone to VR Sickness, a developer would be able to reduce the potential VR Sickness that a user may experience by simply making adjustments to the software only, which would also be applicable for users who may not be able to physically move to input movement commands to a virtual environment. \\
Another technique that is being utilized, but does not necessarily rely on leg movement to help increase a user's presence within a VR environment, involves the concept of leaning to achieve limitation of VR Sickness. Since the vestibular system is one of the key components that regulates a body's balance, by leaning in a particular direction the user can effectively trick their vestibular system into believing they are to move in the direction they are leaning toward, as well as registering this movement as the appropriate movement command within the virtual environment. For example, leaning forward would move you forward, while leaning sideways or backward would be interpreted as sideways and backwards movement commands, respectively\cite{article5cite}. \\
Some approaches to preventing VR Sickness did not involve modifying the software or hardware of a VR configuration at all, like the EmeTerm: an anti-nausea wristband designed to prevent and relieve nausea and vomiting induced by motion sickness, and other such related sicknesses. The wristband is designed to inhibit transmission of a signal to the brain which invokes nausea and vomiting by emitting a low frequency pulse\cite{article2cite}. Since VR Sickness is similar in both cause and effect to motion sickness, this wristband might be sold as a possible treatment and prevention option for users of VR systems. \\ \\

\textbf{3 - Summary and Review} \\
Amongst these articles, various trends became apparent: by using available methods to trick the body's vestibular system, a user's risk of experiencing VR Sickness can be diminished or eliminated entirely. This appears to be the most effective method, but other techniques are also available, such as modifying a user's Field Of View during movement. Most importantly, this shows that there isn't only one type of adjustment that can be made to counter VR Sickness; modifying current hardware or the inclusion of new hardware, as well as employing graphical techniques to the software itself, have both proven to be effective countermeasures. Other minor adjustments that weren't previously mentioned are also known, such as ensuring that the headset display refreshes at least 75 frames per second\cite{article1cite}, maintaining a base Field Of View of around 90 degrees\cite{article3cite}, and increasing user presence with illusions such as a virtual nose in the user's peripheral vision\cite{article3cite}, all of which helped reduce a user's risk of VR Sickness to some extent. However, adjustments can also potentially make a user more prone to VR Sickness if not done correctly, so developers must be mindful when making any adjustments and review if a modification could have a detrimental effect on users. \\
Further possible research could focus on the extents and ranges of which certain adjustments may be deemed beneficial and detrimental, in both aspects of hardware and software. Currently, several companies are debuting new locomotive VR devices, with which a user straps themselves into a rig that positions them over a type of treadmill or other similar designs, which enables a user to utilize a natural walking motion in order to move within the virtual environment. This is not only beneficial in increasing a user's presence and aligning their vestibular system with their visual perception, but it will also assist in immersion as the user won't feel as though they are limited by the physical space that they are using the VR setup in, since they will be able to walk (in place) continuously in any direction for as long as they please. However, these systems are still quite new, and further testing will need to be performed to determine how beneficial this new hardware may be in reducing VR sickness, along with tactile sensation and haptic feedback, which are also potential avenues which may help with immersion and presence, and thus reducing the severity of symptoms a user may experience.

\bibliography{articles}
\bibliographystyle{ieeetr}

\end{document}