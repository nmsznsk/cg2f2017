\documentclass{article}
\begin{document}



\title{Virtual Reality for Stroke Rehabilitation}
\author{Nicholas Muszynski}
\date{September 26, 2017}

\maketitle

\textbf{Summary}
\newline


Stroke is caused by a disruption of blood supply to the brain, due to an artery being blocked or bursting. 
After a stroke, affected patients will experience physical impairments, such as weakness or loss of coordination. 
The conventional treatment for this is Repetitive Task-Specific Training, but there are limitations in staffing and hospital stay durations which hinder patient treatments. 
Virtual Reality was considered as a possible alternative to conventional treatment, and testing was done on groups of patients to determine its efficacy. 
For regaining control over upper limb movement, there was a slight improvement in patients using VR instead of conventional therapy, and was shown to be more effective in patients who went through VR therapy within 6 months of when they had a stroke. 
VR patients also showed significant improvement when compared to patients who received no treatment at all. 
There also didn't appear to be any difference between the effectiveness of VR gear across various price ranges. 
However, the evidence for these results were considered low to low-quality, as their sample size of patients was not large enough. 
While the results are promising, further research will need to be done.


\nocite{article1cite}

\bibliography{article1}
\bibliographystyle{ieeetr}

\end{document}